\documentclass[a4paper,12pt]{article}
\usepackage[utf8]{inputenc}
\usepackage[T1]{fontenc}
\usepackage[spanish]{babel}
\spanishdecimal{.}
\usepackage{csquotes}
\usepackage{anysize}
\usepackage{graphicx}
\marginsize{25mm}{25mm}{25mm}{25mm}

\title{Notas CGIDGAV}
\author{Daniel Maldonado}
\date{}

\begin{document}
{\scshape\bfseries \maketitle}

{\noindent\scshape\bfseries Vinculaciones a proceso}

Hay varios pasos para abrir una carpeta.
De inicio hay aperturas con detenido (en flagrancia) y sin detenido.
También hay carpetas iniciadas por la propia fiscalía responsable y radicadas desde otras fiscalías.
Por último existen reingresos: carpetas que pasaron a una parte más adelante del proceso pero que fueron regresadas por alguna razón.

Después del Inicio existe una investigación inicial, después de la cual se da una Determinación.
Hay varios tipos de determinaciones:
\begin{enumerate}
    \item Judicializar (Ejercicio de Acción Penal): se vincula a proceso.
    \item Criterios de Oportunidad: Dada la naturaleza del delito puede elegirse no investigar (robo de gansito).
	Sin embargo, nunca se realiza en la ciudad debido a que se ve mal que la autoridad decida no investigar un delito.
	En lugar de ello se pasan al Archivo Temporal.
    \item Abstención de investigar.
    \item Acuerdo Reparatorio. Entre la víctima y el imputado se llega a un acuerdo que no derive en vinculación a proceso.
	Por ejemplo, se puede imponer al imputado un curso de masculinidad, más el pago de los daños, más una orden de restricción.
    \item Suspensión Condicional de Proceso. Tampoco se suele dar porque se ve mal.
    \item No Ejercicio de Acción Penal. Al determinarse NEAP y Archivo Temporal las carpetas van a CAMPAP, que las acepta por cuotas. Cuando llegan más carpetas de las que aceptan, éstas se envían de vuelta a determinación como reingreso.
    \item Archivo Temporal. Carpetas que se determinan como estancadas (e.g., por falta de evidencia, no regreso de la víctima) pasan al archivo temporal a esperar el día en que se puedan abrir nuevamente.
	Rara vez sucede.
    \item Incompetencia. Cuando un caso llega a la fiscalía incorrecta, ésta se declara como incompetente y lo transfiere a la correcta.
	Por ejemplo, si llega un delito sexual contra un menor a la FIDS, ésta la transfiere a FIDCANNA.
	En teoría hay un trámite de baja en la fiscalía incompetente, y uno de alta en la competente, pero no es inusual que se pierdan en el inter.
    \item No determinada. Su estado mientras se toma la decisión.
\end{enumerate}

Actualmente solo se reportan las vinculaciones a proceso como medida de productividad, pero eso invisibiliza el grueso del trabajo realizado por las fiscalías.
Alberto pretende cambiar eso pidiendo que se reporten todas las determinaciones.
Nos interesan lo No Ejercicios de Acción Penal, el Archivo Temporal, Incompetencias y Judicializaciones.
Interesa, también, que cuando no haya vinculaciones se reporte una justificación.


\end{document}
